\documentclass[12pt]{article}
\usepackage[utf8]{inputenc}
\usepackage{geometry}
\geometry{letterpaper, left=2.4cm, right=2.4cm, top=2.4cm, bottom=2.4cm}
\usepackage{titlesec}
\usepackage{setspace}
\usepackage{url}

\titleformat{\section}{\bfseries\large}{\thesection}{1em}{}
\titleformat{\subsection}{\bfseries}{\thesubsection}{1em}{}
\titlespacing*{\section}{0pt}{\baselineskip}{0.5\baselineskip}
\titlespacing*{\subsection}{0pt}{\baselineskip}{0.5\baselineskip}

\setlength{\parindent}{0pt} % No indentation
\setlength{\parskip}{1em} % Space between paragraphs

\begin{document}

\title{\textbf{NORMAS APA Y METODOLOGÍA DE LA INVESTIGACIÓN}}
\author{\textit{J. A. Gomez Ramos}\\
\textit{7690-15-16441, Universidad Mariano Gálvez}\\
\textit{Seminario de tecnologias de informacion}\\
\textit{jgomezr11@miumg.edu.gt}}
\date{}
\maketitle

\section*{Resumen}
Las Normas APA y la Metodología de la Investigación son fundamentales para la redacción y presentación de trabajos académicos. Este artículo ofrece una visión detallada de las Normas APA 7ª edición y su importancia en la escritura académica, así como de la Metodología de la Investigación, que guía el proceso desde la formulación de preguntas hasta la interpretación de resultados. Ambas herramientas son esenciales para asegurar la claridad, coherencia y validez de los trabajos académicos.

\section*{Palabras clave}
Normas APA, Metodología de la Investigación, Escritura Académica, Validación de Resultados, Integridad Académica

\section*{Desarrollo del tema}

Las Normas APA, desarrolladas por la American Psychological Association, son un conjunto de reglas y directrices esenciales para la redacción y presentación de documentos científicos y académicos. Estas normas se han actualizado a lo largo del tiempo, y en su séptima edición, se centran en garantizar la claridad de la comunicación, la consistencia en la presentación de la información y el reconocimiento adecuado de las fuentes utilizadas. Este marco normativo es fundamental para mantener la calidad y la integridad de la producción académica.

En cuanto al formato general del documento, las Normas APA especifican que los márgenes deben ser de 1 pulgada (2.54 cm) en todos los lados del papel. El tipo de fuente recomendado es Times New Roman de 12 puntos, aunque también se permiten otras fuentes como Calibri 11, Arial 11 y Lucida Sans Unicode 10. El interlineado debe ser de doble espacio en todo el documento, sin excepciones, lo que facilita la lectura y la revisión del contenido. La alineación del texto debe ser justificada a la izquierda y sin división de palabras al final de las líneas, para asegurar un aspecto limpio y profesional. Además, la primera línea de cada párrafo debe tener una sangría de 0.5 pulgadas (1.27 cm), lo que ayuda a distinguir claramente el comienzo de cada nuevo párrafo.

La estructura del trabajo según las Normas APA incluye varios elementos clave. La página de título debe contener el título del trabajo, el nombre del autor, la afiliación institucional, el curso, el nombre del instructor y la fecha de entrega. Este formato proporciona una introducción clara y organizada del documento. Además, se requiere un resumen que ofrezca una visión general del contenido del documento en 150-250 palabras, seguido de palabras clave que faciliten la búsqueda y clasificación del trabajo en bases de datos académicas. El cuerpo del trabajo se divide en secciones como introducción, método, resultados, discusión y conclusiones, lo que permite una presentación estructurada y lógica de la investigación. La lista de referencias al final del documento debe incluir todas las fuentes citadas, ordenadas alfabéticamente por el apellido del autor, asegurando la correcta atribución de las obras utilizadas en la investigación.

Las citaciones en el texto utilizan el formato autor-fecha, en el cual se menciona el apellido del autor y el año de publicación, por ejemplo, (Smith, 2020). Para citas directas de menos de 40 palabras, estas deben ser incluidas entre comillas dentro del párrafo; para citas de 40 palabras o más, deben presentarse en un bloque independiente sin comillas y con sangría de 0.5 pulgadas. Este formato asegura que las fuentes sean correctamente acreditadas y que los lectores puedan verificar la información original. Las paráfrasis también deben citar al autor y el año de la obra, aunque no se utilice una cita directa, para evitar el plagio y mantener la integridad académica.

La lista de referencias debe seguir un formato específico para diferentes tipos de fuentes. Por ejemplo, para libros se debe utilizar el siguiente formato: Apellido, Inicial del nombre. (Año). Título del libro en cursiva. Editorial. Para artículos de revista: Apellido, Inicial del nombre. (Año). Título del artículo. Título de la Revista en cursiva, volumen(número), rango de páginas. Y para fuentes electrónicas: Apellido, Inicial del nombre. (Año). Título del trabajo. Nombre del sitio web. URL. Este formato estandarizado facilita la localización y verificación de las fuentes utilizadas.

La metodología de la investigación es el conjunto de métodos y técnicas que guían el proceso de investigación científica, desde la formulación de preguntas hasta la interpretación de resultados. Este proceso incluye una serie de pasos sistemáticos diseñados para garantizar la validez y fiabilidad de los hallazgos, y es fundamental para cualquier investigación académica o científica.

El primer paso en la metodología de la investigación es la definición del problema de investigación. Este paso implica identificar claramente el problema o la pregunta de investigación que se pretende abordar. Una vez definido el problema, es necesario realizar una revisión exhaustiva de la literatura existente para contextualizar el problema y establecer una base teórica sólida. Esta revisión ayuda a identificar lagunas en el conocimiento actual y a formular hipótesis relevantes.

El diseño de la investigación es otro componente crucial y puede ser cuantitativo, cualitativo o mixto. El enfoque cuantitativo se basa en la medición numérica y el análisis estadístico de los datos. Este enfoque es útil para investigar relaciones causales y hacer generalizaciones basadas en muestras representativas. Por otro lado, el enfoque cualitativo se centra en la comprensión de fenómenos a través de datos no numéricos, como entrevistas, observaciones y análisis de contenido. Este enfoque es ideal para explorar aspectos complejos y profundos de un fenómeno. El enfoque mixto combina métodos cuantitativos y cualitativos, proporcionando una visión más completa y rica del fenómeno estudiado.

La recolección de datos es una etapa crítica que implica el uso de instrumentos de medición como cuestionarios, encuestas, entrevistas y observaciones. La selección de una muestra representativa de la población es esencial para garantizar la validez de los datos recopilados. Los métodos de muestreo pueden ser probabilísticos o no probabilísticos, dependiendo de los objetivos de la investigación y las características de la población de estudio.

El análisis de datos puede ser cuantitativo o cualitativo, dependiendo del tipo de datos recopilados. El análisis cuantitativo utiliza técnicas estadísticas para analizar los datos y probar las hipótesis formuladas. Herramientas como el software estadístico SPSS o R son comúnmente utilizadas en este tipo de análisis. Por otro lado, el análisis cualitativo implica la codificación de datos, el análisis de contenido y la búsqueda de patrones temáticos. Este tipo de análisis se basa en la interpretación subjetiva de los datos y requiere una comprensión profunda del contexto y los significados subyacentes.

La interpretación y presentación de resultados son etapas en las que se analizan los resultados en el contexto del problema de investigación y la literatura existente. En esta fase, los investigadores deben interpretar los resultados de manera crítica, considerando las limitaciones del estudio y las posibles implicaciones de los hallazgos. Las conclusiones y recomendaciones resumen los hallazgos y proponen acciones basadas en los resultados obtenidos. Estas recomendaciones pueden incluir sugerencias para futuras investigaciones o implicaciones prácticas para la práctica profesional.

Finalmente, la ética en la investigación es fundamental para garantizar la integridad y la responsabilidad del proceso de investigación. Los investigadores deben obtener el consentimiento informado de los participantes, asegurando que comprendan el propósito del estudio y consientan voluntariamente. La confidencialidad de los datos de los participantes debe ser protegida en todo momento, y los investigadores deben ser transparentes en cuanto a cualquier conflicto de interés que pueda influir en los resultados del estudio. La adhesión a principios éticos es crucial para mantener la confianza del público en la investigación científica y para garantizar la validez y la fiabilidad de los resultados.

\section*{Observaciones y comentarios}
Las Normas APA y la Metodología de la Investigación, aunque distintas en su enfoque, son complementarias y esenciales para la producción académica. La correcta aplicación de ambas garantiza la calidad, coherencia y validez de los trabajos, contribuyendo al avance del conocimiento y a la profesionalización de los investigadores.

\section*{Conclusiones}
1. Las Normas APA aseguran una presentación clara y coherente de la información académica.
2. La Metodología de la Investigación guía el proceso investigativo de manera sistemática y rigurosa.
3. Ambas herramientas son esenciales para la calidad y la integridad de la producción académica.
4. La ética en la investigación es fundamental para mantener la confianza del público y la validez de los resultados.
5. La combinación de Normas APA y una sólida Metodología de la Investigación proporciona una base sólida para el avance del conocimiento científico.

\section*{Bibliografía}
Rivas, A. (2024, 25 mayo). Normas APA con plantilla y generador 2024 - Séptima edición. Normas APA. https://normasapa.in/\\
Ortega, C. (2023, 8 septiembre). ¿Qué es la metodología de la investigación? QuestionPro. https://www.questionpro.com/blog/es/metodologia-de-la-investigacion/

\end{document}
